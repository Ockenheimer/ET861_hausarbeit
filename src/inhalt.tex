%!TEX root = ../konzeptpapier.tex
%!TEX spellcheck = de_DE
% Hauptmenuepunkt

%Hier der Inhalt von Überschrift 2 mit einem Beispielbild // und \citep[nach][12\psq]{ley2004soz} einem Literaturverweis

% label brauch man für verlinkungen, also in diesem fall unnötig
%\label{sec:refUeb1}

\section{EMPATHIZE}\label{EMPATHIZE}
\subsection{Beschreibung des Projektumfeldes}\label{Beschreibung des Projektumfeldes}
%Beschreiben Sie unter diesem Punkt kurz das Umfeld (Ihr Unternehmen/ ihre Einrichtung/ Institution/ Verein) indem das Multimediaprodukt eingesetzt werden soll.
%Welche Besonderheiten sind dort evtl. anzutreffen?
%Gibt es Dinge, die man bei der Gestaltung eines Multimediaproduktes besonders beachten muss?
Der Verfasser der Arbeit ist als Finanzbeamter beim Finanzamt Mainz beschäftigt.
Das Finanzamt ist für die Festsetzung und Erhebung von Steuern auf das Einkommen und dem Umsatz im Rahmen der geltenden Gesetzeslage zuständig. Im Rahmen der Festsetzung werden Erklärungen der Steuerpflichtigen bearbeitet. Diese wurden in den Vorjahren auf Papiervordrucken erstellt. 

Zur Vereinfachung der Bearbeitung der Steuererklärungen wurde 2003 das Projekt ELSTER (\grq{}Elektronische STeuerERklärung\grq{}) ins Leben gerufen. Nachdem in den letzten Jahren ein Desktopclient gepflegt wurde, verlagerte sich der Schwerpunkt der Entwicklung hin zu einer vollständigen Onlinebearbeitung der Steuererklärung. Hierdurch konnten Entwicklungskosten eingespart werden und eine Interoperabilität zwischen verschiedenen Betriebssystemen hergestellt werden.

Für die Onlineanwendung \grq{}Mein Elster\grq{} wird seitens der Finanzverwaltung ab 2019 keine Desktopanwendung mehr angeboten. 

Hier setzt das Konzeptpapier und die Hausarbeit an. Es soll eine Tablet-App erstellt werden, die die Registrierung bei \grq{}Mein Elster\grq{} erleichtert und bei der Erstellung der Steuererklärung unterstützt.




\subsection{Beschreibung der Zielgruppe}\label{Beschreibung der Zielgruppe}
%Hier sollen Sie Ihre Zielgruppe beschreiben.
%Welche Altersgruppe umfasst ihre Zielgruppe?
%Liegen Einschränkungen vor?
%Besondere Interessen oder Ansprüche?
%Usw
Die App richtet sich an Steuerpflichtige die Ihre Steuererklärung über \grq{}Mein Elster\grq{} erstellen wollen bzw. müssen.
Diese sind regelmäßig nicht steuerlich vorgebildet. Eine besondere Einschränkung auf ein bestimmtes Alter oder eine sonstige Gruppierung kann jedoch nicht erfolgen. Einzig zu erwähnen wäre, dass die App sich nicht an Kinder richtet, da diese nicht zuletzt aufgrund fehlender Geschäftsfähigkeit ( vgl. §104 \cite{bgb})


\section{DEFINE}\label{DEFINE}


\subsection{Befragung von potentiellen Nutzern und/oder Kollegen}\label{Befragung von potentiellen Nutzern und/oder Kollegen}
Stellen Sie bitte folgende Fragen an Probanden Ihrer Zielgruppe/ oder falls diese nicht erreichbar sind Ihren Kollegen:

Fällt Ihnen spontan ein digitales Produkt/ Anwendung ein, die Sie ich im {Projektumfeld} wünschen würden?

Welche Abläufe/ Strukturen/ organisatorische Regelungen könnten in Ihrer Meinung nach (im Projektumfeld) verbessert werden?

Welche Ausgabemedien (Tablet, Smartphone, …) nutzen Sie am liebsten?

Kennen Sie eine digitale App/ Anwendung, die Sie sich gerne auch im Kontext {Ihres Projektumfeldes} vorstellen können?

(Gerne können Sie auch weitere Fragen formulieren, die Sie an die Probanden stellen und die Ihnen Inspirationen liefern können. )


\subsection{Ergebnisse der Befragung}\label{Ergebnisse der Befragung}
Stellen Sie hier bitte kurz die Ergebnisse der Befragung dar.

\subsection{Projektziel}\label{Projektziel}
Leiten Sie aus den Ergebnissen der Befragung ein Projektziel ab, welches Sie kurz formulieren

% Seitenumbruch
% \newpage

\section{IDEATE}\label{IDEATE}


\subsection{Produktidee (zur Lösung des Projektziels)}\label{Produktidee (zur Lösung des Projektziels) }
Entwickeln Sie eine Produktidee, diese Sie gerne als Prototyp als Lösungsansatz für das Projektziel umsetzen möchten.

\section{PROTOTYPE}\label{PROTOTYPE}

\subsection{Geplantes Ausgabemedium}\label{Geplantes Ausgabemedium}
Welches Ausgabemedium wollen Sie gerne für Ihren Prototypen nutzen? Mit kurzer Begründung bitte. Beachten Sie dazu bitte sowohl die Ergebnisse Ihrer Benutzerbefragung als auch die Voraussetzungen der Projektumgebung.

\subsection{Geplanter Medieneinsatz}\label{Geplanter Medieneinsatz}
Welche Medien (Audio/ Video/ Bild/ …) planen Sie einzusetzen?


