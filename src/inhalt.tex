%!TEX root = ../konzeptpapier.tex
%!TEX spellcheck = de_DE
% Hauptmenuepunkt

%Hier der Inhalt von Überschrift 2 mit einem Beispielbild // und \citep[nach][12\psq]{ley2004soz} einem Literaturverweis

% label brauch man für verlinkungen, also in diesem fall unnötig
%\label{sec:refUeb1}

\section{EMPATHIZE}\label{EMPATHIZE}
\subsection{Beschreibung des Projektumfeldes}\label{Beschreibung des Projektumfeldes}
%Beschreiben Sie unter diesem Punkt kurz das Umfeld (Ihr Unternehmen/ ihre Einrichtung/ Institution/ Verein) indem das Multimediaprodukt eingesetzt werden soll.
%Welche Besonderheiten sind dort evtl. anzutreffen?
%Gibt es Dinge, die man bei der Gestaltung eines Multimediaproduktes besonders beachten muss?
Der Verfasser der Arbeit ist als Finanzbeamter beim Finanzamt Mainz beschäftigt.
Das Finanzamt ist für die Festsetzung und Erhebung von Steuern auf das Einkommen und dem Umsatz im Rahmen der geltenden Gesetzeslage zuständig. Im Rahmen der Festsetzung werden Erklärungen der Steuerpflichtigen bearbeitet. Diese wurden in den Vorjahren auf Papiervordrucken erstellt. 

Zur Vereinfachung der Bearbeitung der Steuererklärungen wurde 2003 das Projekt ELSTER (\grq{}Elektronische STeuerERklärung\grq{}) ins Leben gerufen. Nachdem in den letzten Jahren ein Desktopclient gepflegt wurde, verlagerte sich der Schwerpunkt der Entwicklung hin zu einer vollständigen Onlinebearbeitung der Steuererklärung. Hierdurch konnten Entwicklungskosten eingespart werden und eine Interoperabilität zwischen verschiedenen Betriebssystemen hergestellt werden.

Für die Onlineanwendung \grq{}Mein Elster\grq{} wird seitens der Finanzverwaltung ab 2019 keine Desktopanwendung mehr angeboten.
Um den durchaus komplizierten Registrierungsprozess zu veranschaulichen wurden Erklärvideos produziert und auf Youtube veröffentlicht. Die Verweise zu diesen Videos werden auf der Webseite aber nicht immer wahrgenommen und genutzt. 

Hier setzt das Konzeptpapier und die Hausarbeit an. Es soll eine Tablet-App erstellt werden, die die Registrierung bei \grq{}Mein Elster\grq{} erleichtert und bei der Erstellung der Steuererklärung unterstützt. Hierbei soll auch eine native Einbindung der Videos genutzt werden.




\subsection{Beschreibung der Zielgruppe}\label{Beschreibung der Zielgruppe}
%Hier sollen Sie Ihre Zielgruppe beschreiben.
%Welche Altersgruppe umfasst ihre Zielgruppe?
%Liegen Einschränkungen vor?
%Besondere Interessen oder Ansprüche?
%Usw
Die App richtet sich an Steuerpflichtige die Ihre Steuererklärung über \grq{}Mein Elster\grq{} erstellen wollen bzw. müssen.
Diese sind regelmäßig nicht steuerlich vorgebildet. Eine besondere Einschränkung auf ein bestimmtes Alter oder eine sonstige Gruppierung kann jedoch nicht erfolgen. Einzig zu erwähnen wäre, dass die App sich nicht an Kinder richtet, da diese nicht zuletzt aufgrund fehlender Geschäftsfähigkeit (vgl. §104 \cite{bgb}) keine Steuererklärungen abgeben können. Somit stellen sich aber auch Anforderungen im Sinne der Barrierefreiheit ein, da auch beeinträchtigte Steuerpflichtige die App benutzen könnten.


\section{DEFINE}\label{DEFINE}


\subsection{Nutzerbefragung}%\label{Befragung von potentiellen Nutzern und/oder Kollegen}
%Stellen Sie bitte folgende Fragen an Probanden Ihrer Zielgruppe/ oder falls diese nicht erreichbar sind Ihren Kollegen:

Eine Befragung im klassischen Sinne wurde nicht durchgeführt. Der Autor ist bei der Finanzverwaltung als Finanzbeamter beschäftigt. Hier fungiert er unter Anderem auch als Elster-Ansprechpartner und führt täglich 5-10 Telefonate mit Steuerpflichtigen, die Probleme bei der Registrierung oder der Erstellung von Steuererklärungen in Elster haben. Die Antworten auf die unten aufgeführten Fragen werden zur besseren Lesbarkeit vom Autoren in persönliche Aussagen überführt.

Es sollen folgende Fragen erörtert werden:
\begin{itemize}
\item Wo sind für Sie die größten Probleme, wenn Sie sich die Registrierung bei \grq{}Mein Elster\grq{} betrachten?

\item Wie könnte der Registrierungsprozess Ihrer Meinung nach verbessert werden?

\item Welche Ausgabemedien (Tablet, Smartphone, …) nutzen Sie am liebsten?

\item Kennen Sie eine digitale App/ Anwendung, die Sie sich gerne auch in diesem Kontext vorstellen können?

%(Gerne können Sie auch weitere Fragen formulieren, die Sie an die Probanden stellen und die Ihnen Inspirationen liefern können. )
\end{itemize}


\subsection{Ergebnisse der Befragung}\label{Ergebnisse der Befragung}

\subsubsection{Probleme bei der Registrierung}
Die Kunden äußerten, dass vor allem die Wartezeit zwischen den Registrierungsschritten zu Problemen führt.Ein Nutzerin sagte "Ach Gott! Bis der Brief kommt, hab ich ja die Hälfte vergessen! Wenn das mal reicht." Tatsächlich ist zwischen den Schritten eine notwendige Wartezeit von 10 Werktagen nötig, da der Brief mit dem Aktivierungscode von der jeweiligen Landesdruckerei erstellt und versandt werden muss. Die E-Mail mit der ID-Nummer für die Zwei-Faktor-Authentifizierung verschwindet in der Zeit in den Untiefen des Mailprogramms.

\subsubsection{Verbesserungsmöglichkeiten}
Die  große Forderung der Kunden ist der Wunsch nach einer Vereinfachung des Prozesses. Sie wünschen sich mehr Begleitung und einfachere Anweisungen statt seitenlanger Erklärungen.

\subsubsection{Beliebtes Eingabegerät}
"Ich hab mir extra für Elster einen PC gekauft. Den Rest mach ich auf meinem iPad." Tatsächlich nutzen vieler der Steuerpflichtigen ein Tablet, da es eine intuitivere Benutzung bietet. Der eingeschränkte Funktionsumfang fällt ihnen dabei kaum auf, da ihre Aktivitäten im Internet sehr überschaubar sind. Hier sei das Zitat eines Frührenters beispielhaft: "Ich schau nach, wie morgen das Wetter wird und ob ich mit meinen Enkeln in den Garten kann. Manchmal les' ich auch die Zeitung. Aber mehr muss nicht sein."

\subsubsection{Beliebte Anwendung}
Wenn man überhaupt von einer \grq{}beliebten\grq{} Anwendung sprechen konnte, wurde die Anwendung ElsterFormular, die bis zum Veranlagunszeitraum 2017 von der Finanzverwaltung kostenfrei zur Verfügung gestellt wird, genannt. Dies wurde damit begründet, dass die Eingaben in einem digitalen Formular durchgeführt werden, das in der Optik den alten Papiervordrucken entsprach. Eine Anwenderin erzählte von dem Papiervordruck, den sie vor 12 Jahren mit Hilfe Ihres damaligen Steuerberaters ausgefüllt hat und seit dieser Zeit zum Abschreiben verwendet.

\subsection{Projektziel}\label{Projektziel}

Die meisten Anrufer wünschen sich eine einfachere Anleitung zur Registrierung bei \grq{}Mein Elster\grq{} wünschen. Ziel des Projekts ist, durch die Unterstützung des Steuerpflichtigen bei der Registrierung die Rückfragen zu reduzieren und damit die Zufriedenheit mit dem Produkt \grq{}Mein Elster\grq{} zu steigern.


% Seitenumbruch
% \newpage

\section{IDEATE}\label{IDEATE}


\subsection{Produktidee (zur Lösung des Projektziels)}\label{Produktidee (zur Lösung des Projektziels) }
%Entwickeln Sie eine Produktidee, diese Sie gerne als Prototyp als Lösungsansatz für das Projektziel umsetzen möchten.
Die App, die nun auf einem Tablet realisiert werden soll, sollte den Fortschritt der Registrierung mitverfolgen und beim Start anzeigen, welchen Schritt man als nächstes durchführen muss, und was dafür zu erledigen ist. Zu jedem Schritt soll das jeweilige Youtube-Video aufgerufen werden können. Eine echte Integration der Registrierung in die App wird nicht angestrebt, sie dient ausschließlich als Nachschlagewerk und Unterstützung.

Es soll möglich sein, mehrere Registrierungen gleichzeitig überwachen zu können. Hierzu sollte eine Benutzerauswahl auf dem Startscreen angeboten werden. Sobald der Nutzer ausgewählt ist, soll der jeweils anstehende Schritt präsentiert werden. Die notwendigen Eingaben in der Webseite sollen klar und einfach erläutert werden. Auch sollen die Eingaben (z.B. zum Benutzernamen) in der App gespeichert und für einen späteren Gebrauch aufrufbar sein.
Auch soll die Möglichkeit bestehen, sich bereits durchgeführte Schritte erneut ansehen zu können, sowie den Registriervorgang vollumfänglich zu betrachten. Der Fokus liegt jedoch immer auf dem aktuell durchzuführenden Schritt. Darüber hinaus soll auch das volle Youtube-Video zur Registrierung anzuwählen sein.


\section{PROTOTYPE}\label{PROTOTYPE}

\subsection{Geplantes Ausgabemedium}\label{Geplantes Ausgabemedium}
%Welches Ausgabemedium wollen Sie gerne für Ihren Prototypen nutzen? Mit kurzer Begründung bitte. Beachten Sie dazu bitte sowohl die Ergebnisse Ihrer Benutzerbefragung als auch die Voraussetzungen der Projektumgebung.
Da bei den Befragten hauptsächlich Tablets zum Einsatz kamen, soll dieses Medium genutzt werden. Dies hat den Vorteil, dass auch verschiedene Mechaniken des jeweiligen Betriebssystems genutzt werden können, um z.B. zu einem gewissen Grad Barrierefreiheit herzustellen. So werden z.B. Schriften schon mit einer vom Nutzer gewählten Größe angezeigt.

\subsection{Geplanter Medieneinsatz}\label{Geplanter Medieneinsatz}
%Welche Medien (Audio/ Video/ Bild/ …) planen Sie einzusetzen?
Es sollen neben der schriftlichen und vereinfachten grafischen Darstellung Einbindungen von Youtube-Videos erfolgen. Diese wurden bereits von der Finanzverwaltung produziert und müssen nur noch dargestellt werden. Als Vorteil der Youtube-Videos ist noch anzubringen, dass, sollten in einem Video mehrere Schritte erläutert werden, das Video ab der jeweiligen Passage abgespielt werden kann. 

Den größten Anteil des Medieneinsatzes hat natürlich die textliche Darstellung. Darüber hinaus soll mit einfachen Grafiken und Piktogrammen gearbeitet werden, um es dem Nutzer ohne viel Überlegen zu ermöglichen, die Funktion der Eingabe zu begreifen. 

Darüber hinaus ist der Einsatz von Youtube-Videos geplant die bereits unter dem Konto \grqq{}Elsteronline\grqq{} verfügbar sind. Die Videos sollen in einzelnen Sequenzen gezeigt werden. Somit soll sichergestellt werden, dass der Nutzer nich tvon der Fülle des gesamten Videos überfordert wird, sondern immer nur die aktuell relevanten Informationen erhält.